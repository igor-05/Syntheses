\documentclass[a4paper]{article}

\usepackage{a4}
\usepackage[utf8]{inputenc}
\usepackage[french]{babel}
\usepackage[T1]{fontenc}
\usepackage{graphicx}
\usepackage{amsmath}
\usepackage{amssymb}
\usepackage{float}

\DeclareUnicodeCharacter{2212}{-}
\author{Igor De Bock}
\date{\today}
\title{Mathématiques}

\begin{document}
    
    \section{Qu'est-ce que les math les groupes}
    % définition
    % au fond math = addition
    % expliquer autres opérations et que division inverse multi
    % attention pas encore utiliser chiffres

%
%So essentially, a 'number' is any of a given well-defined category of objects that follow a given list well-defined logically-consistent rules, which are generally used to model and solve problems. A 'number' in that sense is just a basic building block of a method of problem-solving.

%When the problem is "Can the hunters fight the mammoths", then one way to model that involves having some way of counting, of expressing the size of the groups involved - a 'simple' model, certainly, but still a model which can then be used to solve the problem: describe what 'counting' means as part of your model, count the mammoths, count the hunters, use the model to determine which count is larger. We don't think of it that way explicitly, because 'how to count' is so ingrained as if fundamental... but there is no real guarantee that you can count, unless you specifically are building a model which enables counting - that makes the concept of 'the next number' meaningful.

%And there's no guarantee you can count, because at a certain point, you can't meaningfully say what 'the next number' means. If you're working with the Rationals, despite them being 'countably infinite', you'd be hard-pressed to get a useful answer to "What rational number comes next after 3/4?" - but at least there is a way to define the rationals that permits that question to make sense. When you start looking at the Reals, the idea of 'nextness' loses all meaning entirely. "What real number comes next after the square root of two?" feels like a nonsense question, because 'counting' has lost all meaning, though there is still some sense of 'order' (arranging them in some order from smallest to largest in a consistent way) among the Real numbers. By the time you get to Complex numbers, you no longer even have that sense of ordering any more, let alone 'nextness'; is 1+2i larger or smaller than 2+i in any way that has meaning, even though they clearly aren't equal?

%Ultimately the things we casually call numbers are unreasonably effective when used to model and solve problems, to the point that we enshrine them in some special place of importance; in practice, any system of consistent logically manipulatable objects that can be used to model and solve problems are just as 'number-like' as what we all think of as numbers. With that understanding, it seems trivial that whether numbers 'exist' is no more a meaningful question as to whether 'wind' exists - some underlying phenomena or collection of object exists, and we are using our ability to describe and understand those things to talk about them, the patterns they form, and the interactions they have.

    \section{Le système en base dix}
    % Toutes les bases notation scientifique = entre les deux extrèmes, divisés
    % nombre après "," est toujours des dixième (binaire = demi, quart, huitièmes) https://www.youtube.com/watch?v=p8u_k2LIZyo&t=640s
    % chiffre représente chiffre * dizaine
    % expliquer que donnée = nombres de charactères différents
    % expliquer que convention

    \section{Fractions}
    % expliquer manière de montrer nombres
    % multiplier deux côté expliquer 
    % simplification de fraction
    % addition possible que si même dénominateur expliquer pourquoi
    % inversé pour multiplication
    % comment savoir si fraction = nombre entier

    \section{Exposants, racines et polynomes}
    % simplification de polynomes

    \section{Equations}
    % expliquer équation = proportion pas nombre
    % en conséquence tant qu'on garde proportion égalité reste permet trouver x
    % les modifications classiques qu'on peut faire
    % système d'équation --> équation = autan d'inconnues que d'équations ?

    \section{Les Fonctions}
    % 3 blue 1 brown pour tout donc aussi integrales
    % fonction = polynome peut être mis sous forme équation

    \section{Trigonometrie}
    % pas vraiment histoire de triangle mais de cercles
    % fonction trigonométrique utiliser pour décrire

    \section{Nombres complexes}

    \section{Algèbre linéaire}
    % 3 blue 1 brown
\end{document}